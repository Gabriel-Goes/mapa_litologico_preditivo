% FUNDAMENTAÇÃO TEÓRICA
\begin{center}
\begin{minipage}{0.9\textwidth}
\setlength{\fboxsep}{1cm}% Espaçamento interno do fbox
\fbox{% Cria um box ao redor do texto
\parbox{\textwidth}{% Permite o controle de parágrafo dentro do fbox
\setlength{\parindent}{1.5cm} % Define a identação dos parágrafos
\fontsize{30}{36}\selectfont % Ajusta o tamanho da fonte e o espaçamento de linha
 \textbf{Introdução}\\

\par{
    Enfrentando a crescente complexidade e o vasto volume de dados geocientíficos, o projeto "Da Terra ao Código" surge como uma resposta aos desafios contemporâneos da exploração mineral e do mapeamento geológico. Integrando  as tecnologias de inteligência artificial (IA) com avançadas soluções de banco de dados, como PostgreSQL e PostGIS, construímos um novo paradigma nas geociências. Motivado pela visão transformadora de Bergen et al. (2019), nosso projeto estabelece um fluxo de trabalho contínuo e eficiente que abrange desde a etapa pós-coleta até a análise dos resultados dos modelos de IA. O objetivo principal é maximizar o potencial dos modelos preditivos na geração de percepções inovadoras para a exploração de recursos naturais. Buscando promover uma colaboração multidisciplinar essencial entre cientistas da terra e especialistas em computação, visamos desenvolver uma plataforma dinâmica capaz de produzir e atualizar mapas litológicos preditivos com uma precisão e eficiência.
}

            }% fecha parbox
        }% fecha fbox
\\[10mm]
\end{minipage}
\end{center}
