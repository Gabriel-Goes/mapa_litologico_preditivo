\par{Com o aumento da capacidade de computação das máquinas, redução de custos de hardwares e o contínuo desenvolvimento de algoritmos na área da inteligência artificial (IA), as técnicas de aprendizagem de máquina estão sendo cada vez mais aplicadas nos mais diversos ramos das ciências e da indústria. Nas ciências da terra, as técnicas de IA têm uma enorme gama de aplicações nas mais diversas áreas e, atualmente, ocorre um grande desenvolvimento aplicado à busca de novos depósitos minerais, área esta conhecida como prospecção ou exploração mineral. A IA opera por meio de algoritmos, que são passos computacionais a serem seguidos pela máquina, capazes de reconhecer padrões e classificar dados de forma semi-automática ou automática. Os principais algoritmos de aprendizagem de máquina aplicados nas geociências, com foco na exploração mineral, são o Support Vector Machine, Random Forests, Artificial Neural Networks, Naive Bayes e k-Nearest Neighbors, entre outros. Estas técnicas são capazes de classificar quantidades enormes de dados geoespaciais em diferentes escalas, desde imageamento de lâminas petrográficas até imagens orbitais, passando por perfilagem (logging) e levantamentos aeroportados (Aeronaves ou VANTs). Esta iniciação científica visa a revisão da literatura sobre o tema de machine learning aplicada à mineração, buscando o entendimento de suas técnicas, funcionalidades e aplicabilidades aos empreendimentos brasileiros de pesquisa e exploração mineral, construindo um arcabouço teórico do estado da arte, seguida da produção de um material metodológico didático no formato Jupyter Notebook como um guia dos métodos aplicados e disponibilizado na plataforma Github.}
