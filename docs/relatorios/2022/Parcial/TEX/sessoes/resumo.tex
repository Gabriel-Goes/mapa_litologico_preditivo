\chapter*{Resumo} % (fold)
\label{chap:Resumo}
\paragraph{A grafita é um bem mineral de importância tecnológica emergente com as novas propriedades descobertas nas últimas décadas no ramo da engenharia de nanomateriais, como no emprego de fabricação de baterias elétricas, supercondutores e fibras leves de alta resistência, e com potencial para fabricação de materiais essenciais para a indústria. Estes novos usos têm aumentado a demanda pela commoditie, trazendo assim, a necessidade de descoberta de novos depósitos economicamente viáveis levando em conta sua localização, volume e grau de pureza. Recentemente, as técnicas de aprendizagem de máquina têm aumentado a viabilidade dos projetos de prospecção mineral devido ao seu baixo custo de execução e sua alta capacidade de correlação de inúmeras variáveis simultaneamente. Com isto, neste projeto, pretende-se utilizar algoritmos de inteligência computacional e dados de sensores remotos para identificar padrões entre os atributos geofísicos e suas classes litológicas mineralizantes, bem como de suas ocorrências minerais. Assim, desenvolvendo novos mapas litológicos para confrontar os existentes e mapas prospectivos de minério de grafita no sistema de nappes de Socorro–Guaxupé no nordeste do estado de São Paulo, divisa com Minas Gerais.}

