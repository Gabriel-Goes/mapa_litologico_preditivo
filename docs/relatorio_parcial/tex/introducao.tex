\chapter{Introdução}
\section{Considerações Iniciais}
\par{A grafita, um mineral de relevância crescente no panorama tecnológico e industrial, destaca-se por suas propriedades únicas e aplicações versáteis, desde o desenvolvimento de nanomateriais, como o grafeno, até seu uso em produtos industriais diversos. Este projeto de pesquisa foca na exploração de jazidas de grafita utilizando técnicas avançadas de aprendizado de máquina e sensoriamento remoto, concentrando-se no sistema de nappes de Socorro-Guaxupé.}

\par{Exploramos como os métodos geofísicos, em conjunto com o processamento de dados aerogeofísicos, podem ser utilizados para identificar padrões associados à mineralização de grafita. O objetivo é desenvolver um modelo preditivo robusto que melhore a eficiência da prospecção mineral e a gestão dos recursos naturais.}

\par{Esta pesquisa se insere em um contexto onde a demanda por grafita está em ascensão devido às suas aplicações em tecnologias emergentes. A grafita é amplamente distribuída em diversos tipos de rochas, mas a identificação de depósitos economicamente viáveis é um desafio que requer abordagens inovadoras. A integração de dados geofísicos com técnicas de aprendizado de máquina apresenta uma oportunidade única para abordar essa questão, potencializando a identificação de áreas promissoras para exploração mineral.}
